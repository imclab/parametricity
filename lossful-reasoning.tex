\begin{frame}[fragile]
\frametitle{Lossful Reasoning}
\framesubtitle{Sacrificing efficiency to gain unreliability}
Suppose we encountered the following function definition:
\begin{lstlisting}[style=scala]
def add10(n: Int): Int
\end{lstlisting}
By the type alone, there are {$({2^{32}})^{2^{32}}$} possible implementations
\end{frame}

\begin{frame}[fragile]
\frametitle{Lossful Reasoning}
\framesubtitle{Sacrificing efficiency to gain unreliability}
We might form a suspicion that \lstinline[style=scala]$add10$ adds ten to its argument
\begin{lstlisting}[style=scala]
def `add10`(n: Int): Int
\end{lstlisting}
\begin{tikzpicture}[remember picture,overlay]
\coordinate (aa) at ($(a1)+(7,2.0)$);
\node[note,draw,callout relative pointer={($(aa)-(11.2,-3.7)$)},right] at (aa) {\includegraphics[width=0.2\textwidth]{image/suspicion.jpg}};
\end{tikzpicture}
\end{frame}

\begin{frame}[fragile]
\frametitle{Lossful Reasoning}
\framesubtitle{Sacrificing efficiency to gain unreliability}
So we write some tests:
\begin{lstlisting}[style=scala]
add10(0)        = 10
add10(5)        = 15
add10(-5)       = 5
add10(223)      = 233
add10(5096)     = 5106
add10(2914578)  = 29145588
add10(-2914578) = -29145568
\end{lstlisting}
And conclude, yes, this function adds ten to its argument
\end{frame}

\begin{frame}[fragile]
\frametitle{Lossful Reasoning}
\framesubtitle{Sacrificing efficiency to gain unreliability}
\begin{lstlisting}[style=scala]
def add10(n: Int): Int =
  if(n < 8000000) n + 10
  else n * 7
\end{lstlisting}
Wason Rule Discovery Test, \emph{confirmation bias\cite{gale2002does}}.
\end{frame}

\begin{frame}[fragile]
\frametitle{Lossful Reasoning}
\framesubtitle{Sacrificing efficiency to gain unreliability}
We will just write more tests!
\begin{lstlisting}[style=scala]
add10(18916712)  = 18916722
add10(-18916712) = -18916702
\end{lstlisting}
\ldots or we might come up with some system of apologetics for this shortfall
\begin{itemize}
  \item ``A negligent programmer has misnamed this function"
  \item ``More tests will fix it"
  \item ``Well we can't test everything!"
\end{itemize}
\end{frame}

\begin{frame}[fragile]
\frametitle{Lossful Reasoning}
\framesubtitle{Sacrificing efficiency to gain unreliability}
\begin{center}{We are reinforcing our excess confidence in our belief that we
are being responsible programmers}
\end{center}
\end{frame}

\begin{frame}[fragile]
\frametitle{Lossful Reasoning}
\framesubtitle{Sacrificing efficiency to gain unreliability}
\begin{center}
We aren't
\end{center}
\end{frame}

\begin{frame}[fragile]
\frametitle{Lossful Reasoning}
\framesubtitle{Efficiency}
\begin{center}
Actually, we can do significantly better
\end{center}
\end{frame}

\begin{frame}[fragile]
\frametitle{Lossful Reasoning}
\framesubtitle{Reliability}
\begin{center}
And we can have a machine-checked proof, mitigating the disposition to
our biases
\end{center}
\end{frame}
